\documentclass[10pt,a4paper]{article}
\usepackage[utf8]{inputenc}
\usepackage[T1]{fontenc}
\usepackage[provide=*,french]{babel}
\usepackage[margin=1.5cm,top=1.8cm]{geometry}
\usepackage{booktabs}
\usepackage{array}
\usepackage{multicol}
\usepackage{xcolor}
\usepackage{colortbl}
\usepackage{enumitem}

\definecolor{pos1}{HTML}{1A5276}
\definecolor{pos2}{HTML}{922B21}
\definecolor{pos3}{HTML}{196F3D}
\definecolor{pos4}{HTML}{B7950B}
\definecolor{pos5}{HTML}{6C3483}
\definecolor{lightgray}{HTML}{F2F3F4}

\setlength{\parindent}{0pt}
\pagestyle{plain}

\newcounter{excount}
\setcounter{excount}{0}
\newcommand{\ex}[1]{\stepcounter{excount}\theexcount. & #1 \\[0.3em]}

\begin{document}

\begin{center}
{\LARGE\bfseries Corrigé --- Pronoms Compléments}\\[0.3em]
{\large\color{gray} Les pronoms remplacés sont en couleur}
\end{center}

\vspace{0.8em}

\subsection*{Niveau 1 : Un seul pronom}

\begin{tabular}{rl}
\ex{Je \textcolor{pos2}{\textbf{la}} mange. \quad{\small(la pomme $\rightarrow$ la)}}
\ex{Il \textcolor{pos3}{\textbf{lui}} parle. \quad{\small(à Marie $\rightarrow$ lui)}}
\ex{Nous \textcolor{pos2}{\textbf{les}} regardons. \quad{\small(les films $\rightarrow$ les)}}
\ex{Tu \textcolor{pos4}{\textbf{y}} vas. \quad{\small(à Paris $\rightarrow$ y)}}
\ex{Elle \textcolor{pos5}{\textbf{en}} veut. \quad{\small(du pain $\rightarrow$ en)}}
\ex{Ils \textcolor{pos2}{\textbf{le}} connaissent. \quad{\small(mon frère $\rightarrow$ le)}}
\ex{Vous \textcolor{pos3}{\textbf{leur}} téléphonez. \quad{\small(à vos parents $\rightarrow$ leur)}}
\ex{J'\textcolor{pos4}{\textbf{y}} pense. \quad{\small(à ce problème $\rightarrow$ y)}}
\ex{Elle \textcolor{pos5}{\textbf{en}} a besoin. \quad{\small(d'argent $\rightarrow$ en)}}
\ex{Nous \textcolor{pos4}{\textbf{y}} habitons. \quad{\small(en France $\rightarrow$ y)}}
\end{tabular}

\vspace{0.8em}

\subsection*{Niveau 2 : Deux pronoms}

\begin{tabular}{rl}
\ex{Je \textcolor{pos2}{\textbf{le}} \textcolor{pos3}{\textbf{lui}} donne. \quad{\small(le livre $\rightarrow$ le, à Pierre $\rightarrow$ lui)}}
\ex{Elle \textcolor{pos2}{\textbf{la}} \textcolor{pos3}{\textbf{leur}} envoie. \quad{\small(la lettre $\rightarrow$ la, à ses parents $\rightarrow$ leur)}}
\ex{Il \textcolor{pos2}{\textbf{les}} \textcolor{pos3}{\textbf{lui}} montre. \quad{\small(les photos $\rightarrow$ les, à Marie $\rightarrow$ lui)}}
\ex{Tu \textcolor{pos1}{\textbf{me}} \textcolor{pos2}{\textbf{le}} donnes. \quad{\small(me = me, le cahier $\rightarrow$ le)}}
\ex{Nous \textcolor{pos3}{\textbf{lui}} \textcolor{pos5}{\textbf{en}} achetons. \quad{\small(des fleurs $\rightarrow$ en, à notre mère $\rightarrow$ lui)}}
\ex{Elle \textcolor{pos1}{\textbf{m'}}\textcolor{pos5}{\textbf{en}} parle. \quad{\small(me = m', de son voyage $\rightarrow$ en)}}
\ex{Je \textcolor{pos2}{\textbf{la}} \textcolor{pos3}{\textbf{lui}} prête. \quad{\small(ma voiture $\rightarrow$ la, à mon frère $\rightarrow$ lui)}}
\ex{Il \textcolor{pos2}{\textbf{la}} \textcolor{pos3}{\textbf{leur}} dit. \quad{\small(la vérité $\rightarrow$ la, à ses amis $\rightarrow$ leur)}}
\ex{Tu \textcolor{pos3}{\textbf{lui}} \textcolor{pos5}{\textbf{en}} apportes. \quad{\small(du café $\rightarrow$ en, à ta sœur $\rightarrow$ lui)}}
\ex{Vous \textcolor{pos2}{\textbf{le}} \textcolor{pos3}{\textbf{leur}} expliquez. \quad{\small(le problème $\rightarrow$ le, aux élèves $\rightarrow$ leur)}}
\end{tabular}

\vspace{0.8em}

\subsection*{Niveau 3 : Phrases au passé composé}

{\small\color{gray} Rappel : les pronoms vont avant l'auxiliaire. Accord du participe avec le COD placé avant.}

\vspace{0.3em}

\begin{tabular}{rl}
\ex{Je \textcolor{pos2}{\textbf{le}} \textcolor{pos3}{\textbf{lui}} ai donné. \quad{\small(le livre $\rightarrow$ le, à Paul $\rightarrow$ lui)}}
\ex{Elle \textcolor{pos2}{\textbf{les}} \textcolor{pos3}{\textbf{lui}} a envoyés. \quad{\small(les documents $\rightarrow$ les, à son avocat $\rightarrow$ lui, accord: envoyé\textbf{s})}}
\ex{Il \textcolor{pos1}{\textbf{me}} \textcolor{pos2}{\textbf{l'}}a racontée. \quad{\small(m' = me, son histoire $\rightarrow$ l', accord: raconté\textbf{e})}}
\ex{Nous \textcolor{pos3}{\textbf{leur}} \textcolor{pos5}{\textbf{en}} avons acheté. \quad{\small(des cadeaux $\rightarrow$ en, aux enfants $\rightarrow$ leur, pas d'accord avec en)}}
\ex{Tu \textcolor{pos2}{\textbf{les}} \textcolor{pos4}{\textbf{y}} as mises. \quad{\small(les clés $\rightarrow$ les, sur la table $\rightarrow$ y, accord: mis\textbf{es})}}
\ex{Ils \textcolor{pos3}{\textbf{lui}} \textcolor{pos5}{\textbf{en}} ont parlé. \quad{\small(de ce sujet $\rightarrow$ en, à leur patron $\rightarrow$ lui)}}
\ex{Je \textcolor{pos3}{\textbf{lui}} \textcolor{pos5}{\textbf{en}} ai emprunté. \quad{\small(de l'argent $\rightarrow$ en, à ma sœur $\rightarrow$ lui)}}
\ex{Elle \textcolor{pos2}{\textbf{la}} \textcolor{pos3}{\textbf{leur}} a montrée. \quad{\small(sa bague $\rightarrow$ la, à ses amies $\rightarrow$ leur, accord: montré\textbf{e})}}
\end{tabular}

\vspace{0.8em}

\subsection*{Niveau 4 : Négatif}

{\small\color{gray} Rappel : ne + pronoms + verbe + pas}

\vspace{0.3em}

\begin{tabular}{rl}
\ex{Je ne \textcolor{pos2}{\textbf{le}} \textcolor{pos3}{\textbf{lui}} donne pas.}
\ex{Elle ne \textcolor{pos2}{\textbf{la}} \textcolor{pos3}{\textbf{leur}} a pas dite.}
\ex{Il n'\textcolor{pos5}{\textbf{en}} veut pas.}
\ex{Nous ne \textcolor{pos3}{\textbf{lui}} \textcolor{pos5}{\textbf{en}} avons pas parlé.}
\ex{Tu ne \textcolor{pos1}{\textbf{me}} \textcolor{pos2}{\textbf{les}} as pas montrées.}
\end{tabular}

\vspace{0.8em}

\subsection*{Niveau 5 : Impératif}

{\small\color{gray} Rappel : affirmatif = après le verbe avec trait d'union. Négatif = ordre normal avant le verbe.}

\vspace{0.3em}

\begin{tabular}{rl}
\ex{Donne-\textcolor{pos2}{\textbf{le}}-\textcolor{pos3}{\textbf{lui}} !}
\ex{Ne \textcolor{pos2}{\textbf{le}} \textcolor{pos3}{\textbf{lui}} donne pas !}
\ex{Parle-\textcolor{pos3}{\textbf{leur}}-\textcolor{pos5}{\textbf{en}} !}
\ex{Achète-nous-\textcolor{pos5}{\textbf{en}} ! \quad{\small(pour nous $\rightarrow$ nous, du pain $\rightarrow$ en)}}
\ex{Envoie-\textcolor{pos2}{\textbf{la}}-\textcolor{pos3}{\textbf{lui}} !}
\ex{Ne \textcolor{pos2}{\textbf{les}} \textcolor{pos3}{\textbf{leur}} montre pas !}
\ex{Donne-\textcolor{pos1}{\textbf{m'}}\textcolor{pos5}{\textbf{en}} !}
\end{tabular}

\end{document}
