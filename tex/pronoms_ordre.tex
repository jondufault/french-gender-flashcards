\documentclass[10pt,a4paper]{article}
\usepackage[utf8]{inputenc}
\usepackage[T1]{fontenc}
\usepackage[provide=*,french]{babel}
\usepackage[margin=1.5cm,top=1.8cm]{geometry}
\usepackage{booktabs}
\usepackage{array}
\usepackage{multicol}
\usepackage{xcolor}
\usepackage{colortbl}
\usepackage{enumitem}

\definecolor{pos1}{HTML}{1A5276}
\definecolor{pos2}{HTML}{922B21}
\definecolor{pos3}{HTML}{196F3D}
\definecolor{pos4}{HTML}{B7950B}
\definecolor{pos5}{HTML}{6C3483}
\definecolor{lightgray}{HTML}{F2F3F4}

\setlength{\parindent}{0pt}
\pagestyle{plain}

\newcounter{excount}
\setcounter{excount}{0}
\newcommand{\ex}[1]{\stepcounter{excount}\theexcount. & #1 \\[0.5em]}

\begin{document}

\begin{center}
{\LARGE\bfseries L'Ordre des Pronoms Compléments}\\[0.3em]
{\large\color{gray} Référence + Exercices}
\end{center}

\vspace{0.8em}

% ============================================================
\section*{La Règle}

Les pronoms se placent \textbf{avant le verbe} dans cet ordre fixe :

\vspace{0.5em}
\begin{center}
\renewcommand{\arraystretch}{1.6}
\begin{tabular}{|>{\centering\arraybackslash}p{2.2cm}|>{\centering\arraybackslash}p{2.2cm}|>{\centering\arraybackslash}p{2.2cm}|>{\centering\arraybackslash}p{2.2cm}|>{\centering\arraybackslash}p{2.2cm}|}
\hline
\cellcolor{pos1!15}\textcolor{pos1}{\textbf{1}} & \cellcolor{pos2!15}\textcolor{pos2}{\textbf{2}} & \cellcolor{pos3!15}\textcolor{pos3}{\textbf{3}} & \cellcolor{pos4!15}\textcolor{pos4}{\textbf{4}} & \cellcolor{pos5!15}\textcolor{pos5}{\textbf{5}} \\
\hline
\cellcolor{pos1!15} me / te & \cellcolor{pos2!15} le & \cellcolor{pos3!15} lui & \cellcolor{pos4!15} y & \cellcolor{pos5!15} en \\
\cellcolor{pos1!15} se / nous & \cellcolor{pos2!15} la & \cellcolor{pos3!15} leur & \cellcolor{pos4!15} & \cellcolor{pos5!15} \\
\cellcolor{pos1!15} vous & \cellcolor{pos2!15} les & \cellcolor{pos3!15} & \cellcolor{pos4!15} & \cellcolor{pos5!15} \\
\hline
\end{tabular}
\end{center}

\vspace{0.5em}

\textbf{Exemples de référence :}
\begin{itemize}[itemsep=0.2em, leftmargin=1.5em]
\item Je \textcolor{pos2}{\textbf{le}} \textcolor{pos3}{\textbf{lui}} donne. \quad{\small(I give it to him.)}
\item Il \textcolor{pos1}{\textbf{me}} \textcolor{pos2}{\textbf{les}} envoie. \quad{\small(He sends them to me.)}
\item Elle \textcolor{pos2}{\textbf{le}} \textcolor{pos3}{\textbf{leur}} dit. \quad{\small(She tells it to them.)}
\item Je \textcolor{pos3}{\textbf{leur}} \textcolor{pos5}{\textbf{en}} parle. \quad{\small(I talk to them about it.)}
\item Il \textcolor{pos1}{\textbf{m'}}\textcolor{pos4}{\textbf{y}} a emmené. \quad{\small(He took me there.)}
\item Je \textcolor{pos5}{\textbf{n'en}} veux \textbf{pas}. \quad{\small(I don't want any.)}
\end{itemize}

\vspace{0.3em}

\fcolorbox{black}{lightgray}{%
\begin{minipage}{0.95\textwidth}
\vspace{0.3em}
\textbf{Attention --- impératif affirmatif :} l'ordre change !\\
Les pronoms viennent \textbf{après} le verbe, liés par un trait d'union :\\
Donne-\textcolor{pos2}{\textbf{le}}-\textcolor{pos3}{\textbf{lui}} ! \quad Donne-\textcolor{pos1}{\textbf{m'}}\textcolor{pos5}{\textbf{en}} ! \quad Allez-\textcolor{pos4}{\textbf{y}} ! \quad Parle-\textcolor{pos3}{\textbf{leur}}-\textcolor{pos5}{\textbf{en}} !\\
Mais à l'impératif \textbf{négatif}, l'ordre normal revient :\\
Ne \textcolor{pos2}{\textbf{le}} \textcolor{pos3}{\textbf{lui}} donne pas ! \quad Ne \textcolor{pos1}{\textbf{m'}}\textcolor{pos5}{\textbf{en}} donne pas !
\vspace{0.3em}
\end{minipage}%
}

\vspace{1em}

% ============================================================
\section*{Exercices --- Réécrivez en remplaçant par des pronoms}

{\small Écrivez la phrase complète avec les pronoms dans le bon ordre.}

\vspace{0.5em}

\subsection*{Niveau 1 : Un seul pronom}

\renewcommand{\arraystretch}{1.1}
\begin{tabular}{rl}
\ex{Je mange \textbf{la pomme}. \quad $\rightarrow$ \rule{8cm}{0.4pt}}
\ex{Il parle \textbf{à Marie}. \quad $\rightarrow$ \rule{8cm}{0.4pt}}
\ex{Nous regardons \textbf{les films}. \quad $\rightarrow$ \rule{8cm}{0.4pt}}
\ex{Tu vas \textbf{à Paris}. \quad $\rightarrow$ \rule{8cm}{0.4pt}}
\ex{Elle veut \textbf{du pain}. \quad $\rightarrow$ \rule{8cm}{0.4pt}}
\ex{Ils connaissent \textbf{mon frère}. \quad $\rightarrow$ \rule{8cm}{0.4pt}}
\ex{Vous téléphonez \textbf{à vos parents}. \quad $\rightarrow$ \rule{8cm}{0.4pt}}
\ex{Je pense \textbf{à ce problème}. \quad $\rightarrow$ \rule{8cm}{0.4pt}}
\ex{Elle a besoin \textbf{d'argent}. \quad $\rightarrow$ \rule{8cm}{0.4pt}}
\ex{Nous habitons \textbf{en France}. \quad $\rightarrow$ \rule{8cm}{0.4pt}}
\end{tabular}

\vspace{0.8em}

\subsection*{Niveau 2 : Deux pronoms}

\begin{tabular}{rl}
\ex{Je donne \textbf{le livre} \textbf{à Pierre}. \quad $\rightarrow$ \rule{7cm}{0.4pt}}
\ex{Elle envoie \textbf{la lettre} \textbf{à ses parents}. \quad $\rightarrow$ \rule{7cm}{0.4pt}}
\ex{Il montre \textbf{les photos} \textbf{à Marie}. \quad $\rightarrow$ \rule{7cm}{0.4pt}}
\ex{Tu \textbf{me} donnes \textbf{le cahier}. \quad $\rightarrow$ \rule{7cm}{0.4pt}}
\ex{Nous achetons \textbf{des fleurs} \textbf{à notre mère}. \quad $\rightarrow$ \rule{7cm}{0.4pt}}
\ex{Elle \textbf{me} parle \textbf{de son voyage}. \quad $\rightarrow$ \rule{7cm}{0.4pt}}
\ex{Je prête \textbf{ma voiture} \textbf{à mon frère}. \quad $\rightarrow$ \rule{7cm}{0.4pt}}
\ex{Il dit \textbf{la vérité} \textbf{à ses amis}. \quad $\rightarrow$ \rule{7cm}{0.4pt}}
\ex{Tu apportes \textbf{du café} \textbf{à ta sœur}. \quad $\rightarrow$ \rule{7cm}{0.4pt}}
\ex{Vous expliquez \textbf{le problème} \textbf{aux élèves}. \quad $\rightarrow$ \rule{7cm}{0.4pt}}
\end{tabular}

\newpage

\subsection*{Niveau 3 : Phrases au passé composé}

\begin{tabular}{rl}
\ex{J'ai donné \textbf{le livre} \textbf{à Paul}. \quad $\rightarrow$ \rule{7cm}{0.4pt}}
\ex{Elle a envoyé \textbf{les documents} \textbf{à son avocat}. \quad $\rightarrow$ \rule{7cm}{0.4pt}}
\ex{Il \textbf{m}'a raconté \textbf{son histoire}. \quad $\rightarrow$ \rule{7cm}{0.4pt}}
\ex{Nous avons acheté \textbf{des cadeaux} \textbf{aux enfants}. \quad $\rightarrow$ \rule{7cm}{0.4pt}}
\ex{Tu as mis \textbf{les clés} \textbf{sur la table}. \quad $\rightarrow$ \rule{7cm}{0.4pt}}
\ex{Ils ont parlé \textbf{de ce sujet} \textbf{à leur patron}. \quad $\rightarrow$ \rule{7cm}{0.4pt}}
\ex{J'ai emprunté \textbf{de l'argent} \textbf{à ma sœur}. \quad $\rightarrow$ \rule{7cm}{0.4pt}}
\ex{Elle a montré \textbf{sa bague} \textbf{à ses amies}. \quad $\rightarrow$ \rule{7cm}{0.4pt}}
\end{tabular}

\vspace{0.8em}

\subsection*{Niveau 4 : Négatif}

\begin{tabular}{rl}
\ex{Je ne donne pas \textbf{le livre} \textbf{à Pierre}. \quad $\rightarrow$ \rule{6.5cm}{0.4pt}}
\ex{Elle n'a pas dit \textbf{la vérité} \textbf{à ses parents}. \quad $\rightarrow$ \rule{6.5cm}{0.4pt}}
\ex{Il ne veut pas \textbf{de café}. \quad $\rightarrow$ \rule{6.5cm}{0.4pt}}
\ex{Nous n'avons pas parlé \textbf{de ça} \textbf{à notre prof}. \quad $\rightarrow$ \rule{6.5cm}{0.4pt}}
\ex{Tu ne \textbf{m}'as pas montré \textbf{les photos}. \quad $\rightarrow$ \rule{6.5cm}{0.4pt}}
\end{tabular}

\vspace{0.8em}

\subsection*{Niveau 5 : Impératif}

\begin{tabular}{rl}
\ex{Donne \textbf{le livre} \textbf{à Marie} ! \quad $\rightarrow$ \rule{7cm}{0.4pt}}
\ex{Ne donne pas \textbf{le livre} \textbf{à Marie} ! \quad $\rightarrow$ \rule{7cm}{0.4pt}}
\ex{Parle \textbf{de ton projet} \textbf{à tes parents} ! \quad $\rightarrow$ \rule{7cm}{0.4pt}}
\ex{Achète \textbf{du pain} \textbf{pour nous} ! \quad $\rightarrow$ \rule{7cm}{0.4pt}}
\ex{Envoie \textbf{la lettre} \textbf{à ton frère} ! \quad $\rightarrow$ \rule{7cm}{0.4pt}}
\ex{Ne montre pas \textbf{les photos} \textbf{à tes amis} ! \quad $\rightarrow$ \rule{7cm}{0.4pt}}
\ex{Donne-\textbf{moi} \textbf{de l'eau} ! \quad $\rightarrow$ \rule{7cm}{0.4pt}}
\end{tabular}

\vfill
\begin{center}
{\small\color{gray} Score : \rule{1cm}{0.4pt} / 40}
\end{center}

\end{document}
