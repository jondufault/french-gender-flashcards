\documentclass[10pt,a4paper]{article}
\usepackage[utf8]{inputenc}
\usepackage[T1]{fontenc}
\usepackage[provide=*,french]{babel}
\usepackage[margin=1.5cm,top=1.8cm]{geometry}
\usepackage{booktabs}
\usepackage{array}
\usepackage{multicol}
\usepackage{xcolor}
\usepackage{colortbl}

\definecolor{lightgray}{HTML}{F2F3F4}
\definecolor{accent}{HTML}{1A5276}

\setlength{\parindent}{0pt}
\pagestyle{plain}

\newcounter{excount}
\setcounter{excount}{0}
\newcommand{\ex}[1]{\stepcounter{excount}\theexcount. & #1 \\[0.5em]}

\begin{document}

\begin{center}
{\LARGE\bfseries Les Participes Passés Irréguliers}\\[0.3em]
{\large\color{gray} Référence + Exercices}
\end{center}

\vspace{0.8em}

\section*{Référence --- 60 verbes irréguliers courants}

\begin{multicols}{3}

\subsection*{En \textcolor{accent}{-u}}
\small
\renewcommand{\arraystretch}{1.15}
\begin{tabular}{ll}
\toprule
\textbf{Infinitif} & \textbf{Participe} \\
\midrule
avoir & eu \\
boire & bu \\
connaître & connu \\
courir & couru \\
croire & cru \\
devoir & dû \\
falloir & fallu \\
lire & \textbf{lu} \\
paraître & paru \\
plaire & \textbf{plu} \\
pleuvoir & plu \\
pouvoir & pu \\
recevoir & reçu \\
savoir & su \\
tenir & tenu \\
venir & venu \\
vivre & vécu \\
voir & vu \\
vouloir & voulu \\
\bottomrule
\end{tabular}

\columnbreak

\subsection*{En \textcolor{accent}{-is}}
\small
\begin{tabular}{ll}
\toprule
\textbf{Infinitif} & \textbf{Participe} \\
\midrule
apprendre & appris \\
asseoir & assis \\
comprendre & compris \\
mettre & mis \\
permettre & permis \\
prendre & pris \\
promettre & promis \\
soumettre & soumis \\
surprendre & surpris \\
\bottomrule
\end{tabular}

\vspace{1em}

\subsection*{En \textcolor{accent}{-it}}
\small
\begin{tabular}{ll}
\toprule
\textbf{Infinitif} & \textbf{Participe} \\
\midrule
conduire & conduit \\
construire & construit \\
détruire & détruit \\
dire & dit \\
écrire & écrit \\
interdire & interdit \\
produire & produit \\
traduire & traduit \\
\bottomrule
\end{tabular}

\columnbreak

\subsection*{En \textcolor{accent}{-ert}}
\small
\begin{tabular}{ll}
\toprule
\textbf{Infinitif} & \textbf{Participe} \\
\midrule
couvrir & couvert \\
découvrir & découvert \\
offrir & offert \\
ouvrir & ouvert \\
souffrir & souffert \\
\bottomrule
\end{tabular}

\vspace{1em}

\subsection*{En \textcolor{accent}{-int / -aint}}
\small
\begin{tabular}{ll}
\toprule
\textbf{Infinitif} & \textbf{Participe} \\
\midrule
atteindre & atteint \\
craindre & craint \\
éteindre & éteint \\
joindre & joint \\
peindre & peint \\
plaindre & plaint \\
\bottomrule
\end{tabular}

\vspace{1em}

\subsection*{Autres}
\small
\begin{tabular}{ll}
\toprule
\textbf{Infinitif} & \textbf{Participe} \\
\midrule
être & été \\
faire & fait \\
mourir & mort \\
naître & né \\
rire & ri \\
suivre & suivi \\
\bottomrule
\end{tabular}

\end{multicols}

\vspace{0.5em}

\fcolorbox{black}{lightgray}{%
\begin{minipage}{0.95\textwidth}
\vspace{0.3em}
\textbf{Tes erreurs de la lettre :}\\
\begin{tabular}{lll}
\textcolor{red}{*lit} (lire) & $\rightarrow$ & \textbf{lu} \quad (groupe -u : lire, plaire, croire...) \\
\textcolor{red}{*plaît} (plaire) & $\rightarrow$ & \textbf{plu} \quad (c'est le présent il plaît $\neq$ le participe plu) \\
\textcolor{red}{*expulser} & $\rightarrow$ & \textbf{expulsé} \quad (verbes en -er sont toujours réguliers : -é) \\
\end{tabular}
\vspace{0.3em}
\end{minipage}%
}

\vspace{1em}

% ============================================================
\section*{Exercice A --- Donnez le participe passé}

\begin{multicols}{3}
\small
\begin{tabular}{rl}
1. & avoir : \rule{2.5cm}{0.4pt} \\[0.4em]
2. & être : \rule{2.5cm}{0.4pt} \\[0.4em]
3. & faire : \rule{2.5cm}{0.4pt} \\[0.4em]
4. & dire : \rule{2.5cm}{0.4pt} \\[0.4em]
5. & lire : \rule{2.5cm}{0.4pt} \\[0.4em]
6. & écrire : \rule{2.5cm}{0.4pt} \\[0.4em]
7. & prendre : \rule{2.5cm}{0.4pt} \\[0.4em]
8. & mettre : \rule{2.5cm}{0.4pt} \\[0.4em]
9. & voir : \rule{2.5cm}{0.4pt} \\[0.4em]
10. & croire : \rule{2.5cm}{0.4pt} \\[0.4em]
11. & boire : \rule{2.5cm}{0.4pt} \\[0.4em]
12. & connaître : \rule{2.5cm}{0.4pt} \\[0.4em]
13. & savoir : \rule{2.5cm}{0.4pt} \\[0.4em]
14. & pouvoir : \rule{2.5cm}{0.4pt} \\[0.4em]
15. & vouloir : \rule{2.5cm}{0.4pt} \\[0.4em]
16. & devoir : \rule{2.5cm}{0.4pt} \\[0.4em]
17. & venir : \rule{2.5cm}{0.4pt} \\[0.4em]
18. & tenir : \rule{2.5cm}{0.4pt} \\[0.4em]
19. & vivre : \rule{2.5cm}{0.4pt} \\[0.4em]
20. & suivre : \rule{2.5cm}{0.4pt} \\[0.4em]
\end{tabular}

\begin{tabular}{rl}
21. & mourir : \rule{2.5cm}{0.4pt} \\[0.4em]
22. & naître : \rule{2.5cm}{0.4pt} \\[0.4em]
23. & ouvrir : \rule{2.5cm}{0.4pt} \\[0.4em]
24. & offrir : \rule{2.5cm}{0.4pt} \\[0.4em]
25. & souffrir : \rule{2.5cm}{0.4pt} \\[0.4em]
26. & plaire : \rule{2.5cm}{0.4pt} \\[0.4em]
27. & recevoir : \rule{2.5cm}{0.4pt} \\[0.4em]
28. & conduire : \rule{2.5cm}{0.4pt} \\[0.4em]
29. & traduire : \rule{2.5cm}{0.4pt} \\[0.4em]
30. & construire : \rule{2.5cm}{0.4pt} \\[0.4em]
31. & comprendre : \rule{2.5cm}{0.4pt} \\[0.4em]
32. & apprendre : \rule{2.5cm}{0.4pt} \\[0.4em]
33. & promettre : \rule{2.5cm}{0.4pt} \\[0.4em]
34. & permettre : \rule{2.5cm}{0.4pt} \\[0.4em]
35. & peindre : \rule{2.5cm}{0.4pt} \\[0.4em]
36. & craindre : \rule{2.5cm}{0.4pt} \\[0.4em]
37. & éteindre : \rule{2.5cm}{0.4pt} \\[0.4em]
38. & joindre : \rule{2.5cm}{0.4pt} \\[0.4em]
39. & courir : \rule{2.5cm}{0.4pt} \\[0.4em]
40. & rire : \rule{2.5cm}{0.4pt} \\[0.4em]
\end{tabular}

\begin{tabular}{rl}
41. & découvrir : \rule{2.5cm}{0.4pt} \\[0.4em]
42. & surprendre : \rule{2.5cm}{0.4pt} \\[0.4em]
43. & interdire : \rule{2.5cm}{0.4pt} \\[0.4em]
44. & produire : \rule{2.5cm}{0.4pt} \\[0.4em]
45. & détruire : \rule{2.5cm}{0.4pt} \\[0.4em]
46. & paraître : \rule{2.5cm}{0.4pt} \\[0.4em]
47. & atteindre : \rule{2.5cm}{0.4pt} \\[0.4em]
48. & soumettre : \rule{2.5cm}{0.4pt} \\[0.4em]
49. & plaindre : \rule{2.5cm}{0.4pt} \\[0.4em]
50. & falloir : \rule{2.5cm}{0.4pt} \\[0.4em]
\end{tabular}
\end{multicols}

\vspace{0.8em}

\section*{Exercice B --- Complétez avec le participe passé}

{\small Attention à l'accord avec le COD placé avant le verbe.}

\vspace{0.3em}

\begin{tabular}{rl}
\ex{Elle a \rule{2cm}{0.4pt} (lire) tous les livres que je lui ai \rule{2cm}{0.4pt} (offrir).}
\ex{La lettre que j'ai \rule{2cm}{0.4pt} (écrire) n'a jamais été \rule{2cm}{0.4pt} (recevoir).}
\ex{Les enfants ont \rule{2cm}{0.4pt} (boire) le jus que leur mère avait \rule{2cm}{0.4pt} (faire).}
\ex{La porte était \rule{2cm}{0.4pt} (ouvrir) et la lumière \rule{2cm}{0.4pt} (éteindre).}
\ex{Elle est \rule{2cm}{0.4pt} (naître) en France et elle a \rule{2cm}{0.4pt} (vivre) à Paris.}
\ex{La décision qu'ils ont \rule{2cm}{0.4pt} (prendre) m'a \rule{2cm}{0.4pt} (surprendre).}
\ex{J'ai \rule{2cm}{0.4pt} (devoir) partir parce que je n'ai pas \rule{2cm}{0.4pt} (pouvoir) rester.}
\ex{Les tableaux qu'il a \rule{2cm}{0.4pt} (peindre) ont \rule{2cm}{0.4pt} (plaire) à tout le monde.}
\ex{Est-ce que tu as \rule{2cm}{0.4pt} (comprendre) ce qu'il a \rule{2cm}{0.4pt} (dire) ?}
\ex{Nous avons \rule{2cm}{0.4pt} (suivre) le cours qu'on nous avait \rule{2cm}{0.4pt} (promettre).}
\end{tabular}

\vfill
\begin{center}
{\small\color{gray} Score A : \rule{1cm}{0.4pt} / 50 \qquad Score B : \rule{1cm}{0.4pt} / 20}
\end{center}

\end{document}
