\documentclass[10pt,a4paper]{article}
\usepackage[utf8]{inputenc}
\usepackage[T1]{fontenc}
\usepackage[provide=*,french]{babel}
\usepackage[top=1.5cm,bottom=1.5cm,left=1.5cm,right=1.5cm]{geometry}
\usepackage{booktabs}
\usepackage{array}
\usepackage{xcolor}
\usepackage{colortbl}
\usepackage{multicol}
\usepackage{enumitem}

\definecolor{masccolor}{HTML}{1A5276}
\definecolor{femcolor}{HTML}{922B21}
\definecolor{lightmasc}{HTML}{D6EAF8}
\definecolor{lightfem}{HTML}{FADBD8}
\definecolor{lightgray}{HTML}{F2F3F4}
\definecolor{goldbox}{HTML}{FEF9E7}
\definecolor{goldborder}{HTML}{D4AC0D}

\setlength{\parindent}{0pt}
\pagestyle{empty}

\newcommand{\mascheader}[1]{\cellcolor{lightmasc}\textbf{#1}}
\newcommand{\femheader}[1]{\cellcolor{lightfem}\textbf{#1}}

\begin{document}

\begin{center}
{\LARGE\bfseries Règles du Genre des Noms Français}\\[0.2em]
{\large\color{gray} Aide-mémoire pour la préparation du DELF}
\end{center}

\vspace{0.4em}

%% ============================================
%% THE TWO BASE RULES
%% ============================================
\fcolorbox{goldborder}{goldbox}{%
\begin{minipage}{0.96\textwidth}
\vspace{0.4em}
\centerline{\large\bfseries Les Règles de Base}
\vspace{0.3em}

\textcolor{femcolor}{\textbf{Règle 1 :}} Les noms en \textbf{-tion}, \textbf{-sion} et \textbf{-aison} sont \textbf{toujours} \textcolor{femcolor}{\textbf{féminins}}.\\
{\small la situa\textbf{tion}, la déci\textbf{sion}, la ques\textbf{tion}, l'atten\textbf{tion}, la mai\textbf{son}, la rai\textbf{son}, la sai\textbf{son}}

\vspace{0.4em}

\textcolor{masccolor}{\textbf{Règle 2 :}} Les autres noms qui finissent par une \textbf{consonne écrite} sont presque toujours \textcolor{masccolor}{\textbf{masculins}}.\\
{\small le temp\textbf{s}, le jou\textbf{r}, le pay\textbf{s}, le poin\textbf{t}, le droi\textbf{t}, le ca\textbf{s}, le mo\textbf{t}, le corp\textbf{s}, le moi\textbf{s}, l'enfan\textbf{t}}

\vspace{0.2em}
{\small \textbf{Aucune exception :} les terminaisons -ment, -oir, -al, -ail, -et, -ard sont toujours masculines.}

\vspace{0.2em}
{\small \textbf{Cas particulier} \textit{-eur} : souvent \textcolor{masccolor}{masc.} (le bonheur, l'honneur), mais \textcolor{femcolor}{fém.} pour :
la fleur, la peur, la douleur, la chaleur, la couleur, la valeur, l'erreur}

\vspace{0.2em}
{\small \textcolor{femcolor}{\textbf{Autres exceptions féminines :}} la main, la nuit, la fois, la fin, la part, la mort, la dent,
la faim, la soif, la croix, la voix, la clef, la forêt, la mer, la leçon, la façon, la chanson}
\vspace{0.3em}
\end{minipage}%
}

\vspace{0.5em}

{\small\textit{Pour les autres noms (terminaison en voyelle ou en -e), la terminaison spécifique détermine le genre :}}

\vspace{0.4em}

%% ============================================
%% MASCULINE SPECIFIC ENDINGS
%% ============================================
\section*{\color{masccolor} Terminaisons Masculines}

\renewcommand{\arraystretch}{1.15}
\begin{tabular}{>{\bfseries}p{2.5cm} p{6cm} p{6cm}}
\toprule
\mascheader{Terminaison} & \mascheader{Exemples} & \mascheader{Exceptions courantes} \\
\midrule
-age & le voyage, le message, le ménage & la plage, la page, l'image, la cage \\
\rowcolor{lightgray}
-isme & le tourisme, le capitalisme, le réalisme & (aucune) \\
-eau & le bateau, le chapeau, le bureau & la peau, l'eau \\
\rowcolor{lightgray}
-ège & le collège, le piège, le privilège & (aucune) \\
-phone & le téléphone, le microphone & (aucune) \\
\rowcolor{lightgray}
-scope & le microscope, le télescope & (aucune) \\
-acle & le spectacle, l'obstacle, le miracle & (aucune) \\
\rowcolor{lightgray}
-ange & le mélange, l'ange, le change & la vidange, la louange, la grange \\
-o & le vélo, le métro, le piano & la moto, la radio, la photo (abrév.\ fém.) \\
\rowcolor{lightgray}
grec / latin en -e & le problème, le système, le thème & (aucune) \\
\bottomrule
\end{tabular}

\vspace{0.8em}

%% ============================================
%% FEMININE SPECIFIC ENDINGS
%% ============================================
\section*{\color{femcolor} Terminaisons Féminines}

\begin{tabular}{>{\bfseries}p{2.5cm} p{6cm} p{6cm}}
\toprule
\femheader{Terminaison} & \femheader{Exemples} & \femheader{Exceptions courantes} \\
\midrule
-té & la liberté, la société, la santé & le côté, le comité, le traité \\
\rowcolor{lightgray}
-ée & la journée, l'idée, la pensée & le musée, le lycée, le trophée \\
-ure & la voiture, la nature, la culture & le murmure \\
\rowcolor{lightgray}
-ance / -ence & la France, la patience, la violence & le silence \\
-ie & la vie, la philosophie, la maladie & le génie, l'incendie \\
\rowcolor{lightgray}
-oire & l'histoire, la mémoire, la victoire & le territoire, le laboratoire \\
-ose & la chose, la rose, la dose, la prose & (aucune) \\
\rowcolor{lightgray}
-ace & la place, la glace, la surface & l'espace (masc.) \\
-ue & la rue, la vue, la statue, la revue & (rare) \\
\rowcolor{lightgray}
-ole & l'école, la parole, la console & le rôle, le contrôle \\
-onne & la personne, la couronne, la colonne & (aucune) \\
\rowcolor{lightgray}
-eure & l'heure, la demeure & (aucune) \\
-ise & la surprise, la crise, l'entreprise & (aucune) \\
\rowcolor{lightgray}
-ine & la machine, la cuisine, la routine & le magazine \\
-ude & l'attitude, l'étude, la certitude & (aucune) \\
\rowcolor{lightgray}
-esse & la jeunesse, la richesse, la vitesse & (aucune) \\
-aison & la livraison, la comparaison & (aucune) \\
\rowcolor{lightgray}
-ade & la salade, la promenade, la façade & le stade \\
-ère & la rivière, la lumière, la frontière & le cimetière, le caractère \\
\rowcolor{lightgray}
-ette & la fourchette, la cigarette, la dette & le squelette \\
-ille & la ville, la fille, la famille & le gorille \\
\rowcolor{lightgray}
-elle & la demoiselle, la poubelle, la gazelle & (aucune) \\
cons.\ double + e & la terre, la guerre, la femme & l'homme, le gramme \\
\bottomrule
\end{tabular}

\vspace{0.6em}

%% ============================================
%% SEMANTIC RULES
%% ============================================
\section*{Règles Sémantiques}

\begin{multicols}{2}
\begin{enumerate}[leftmargin=1.5em, itemsep=0.2em]
\item \textbf{Langues} $\rightarrow$ \textcolor{masccolor}{\textbf{masc.}} : le français, l'anglais
\item \textbf{Jours et saisons} $\rightarrow$ \textcolor{masccolor}{\textbf{masc.}} : le lundi, le printemps
\item \textbf{Arbres} $\rightarrow$ \textcolor{masccolor}{\textbf{masc.}} : le chêne, le sapin
\item \textbf{Fruits} $\rightarrow$ souvent \textcolor{femcolor}{\textbf{fém.}} : la pomme, la poire
\end{enumerate}
\end{multicols}

\vspace{0.4em}

%% ============================================
%% BLANK ERROR BOX
%% ============================================
\fcolorbox{black}{lightgray}{%
\begin{minipage}{0.96\textwidth}
\vspace{0.3em}
\centerline{\large\bfseries Mes erreurs fréquentes}
\vspace{0.2em}
\renewcommand{\arraystretch}{2.0}
\begin{tabular}{p{0.45\textwidth} p{0.45\textwidth}}
\dotfill & \dotfill \\
\dotfill & \dotfill \\
\dotfill & \dotfill \\
\end{tabular}
\vspace{0.2em}
\end{minipage}%
}

\end{document}
